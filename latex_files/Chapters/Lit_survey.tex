% Some LaTeX commands I define for my own nomenclature.
% If you have to, it's better to change nomenclature once here than in a 
% million places throughout your thesis!
\newcommand{\package}[1]{\textbf{#1}} % package names in bold text
\newcommand{\cmmd}[1]{\textbackslash\texttt{#1}} % command name in tt font 


%======================================================================
\chapter{Background Analysis}
%======================================================================

\section{Object Detection in Google Earth Images Using Deep Convolutional Neural Networks}
The paper\cite{one} proposes a deep convolutional neural network (CNN) approach for object detection in Google Earth images, particularly for detecting buildings. The paper highlights the limitations of existing methods that rely on manual feature extraction and classification, and proposes that deep learning techniques can improve building detection in Google Earth images. The authors provide details of their methodology, which involves training a CNN model on a large dataset of annotated Google Earth images, and discuss the steps involved in preparing the dataset and training the CNN model. They then evaluate the performance of the model using a test set of Google Earth images and compare it to existing methods. The results show that the proposed method outperforms existing methods in terms of accuracy and scalability, demonstrating the potential of deep learning techniques for object detection in Google Earth imagery.

\section{Automated detection of urban change using Google Earth imagery and machine learning}
The paper \cite{two} provides a comprehensive review of traditional and machine learning-based approaches for detecting urban changes using Google Earth imagery. The authors discuss the advantages and limitations of various techniques, including image differencing, object-based change detection, and supervised and unsupervised machine learning methods. The authors also highlight the potential of deep learning-based methods, such as convolutional neural networks (CNNs), for urban change detection due to their ability to learn complex features from the images.

\section{A Deep Learning Approach to Automatic Building Detection in Google Earth Imagery}
The paper \cite{three} proposed a method on a test dataset of Google Earth images and compared its performance with other state-of-the-art methods. The results showed that their method outperformed other methods in terms of accuracy and processing time. The paper concludes that the proposed method has the potential to be used for various applications, including urban planning, disaster management, and environmental monitoring. The authors suggest that future research could focus on improving the accuracy of the model by incorporating additional data sources, such as LiDAR and multi-spectral imagery, and exploring the use of transfer learning to adapt the model to different geographical regions.



\section{Object Detection from Satellite Imagery using Deep Learning Techniques}

The paper \cite{four} presents a method for object detection from satellite imagery using deep learning techniques. Object detection from satellite imagery faces several challenges such as lighting variations, cloud cover, and variations in object size and orientation. To overcome these challenges, the paper proposes a deep learning model based on the Faster R-CNN architecture, which is trained on a large dataset of satellite imagery. The model uses a convolutional neural network (CNN) to extract features from the input image, followed by region proposal generation and object classification. The performance of the proposed method was evaluated on a publicly available dataset of satellite imagery, and the results showed that the method achieved higher accuracy and faster processing time compared to other methods. 