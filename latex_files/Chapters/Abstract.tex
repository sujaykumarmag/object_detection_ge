% A B S T R A C T
% ---------------

\begin{center}\textbf{Abstract}\end{center}
Object detection using Google Earth is a computer vision technology that utilizes state-of-the-art deep learning models to automatically identify and locate objects of interest within satellite or aerial imagery provided by Google Earth. The underlying algorithmic pipeline typically involves multi-scale feature extraction, object proposal generation, and object classification and localization. This process is often executed on high-performance computing infrastructure, such as GPU clusters, to ensure real-time or near-real-time performance.\\

The detection and classification of objects within satellite imagery involves several challenges, such as occlusion, varying lighting conditions, and complex background scenes. To overcome these challenges, deep learning models, such as convolutional neural networks (CNNs), are trained on large-scale datasets that contain annotated examples of different object categories. These models are then fine-tuned on specific tasks, such as building detection or road extraction, to improve their accuracy and efficiency.\\

Object detection using Google Earth has a wide range of practical applications, including urban planning, environmental monitoring, land use mapping, and disaster response. For example, this technology can be used to detect changes in urban infrastructure, such as new road construction or building developments, or to identify areas affected by natural disasters, such as floods or wildfires. Moreover, it can also support conservation efforts by enabling the monitoring of biodiversity and deforestation rates, as well as identifying illegal activities, such as poaching or logging.\\

The proposed project aims to develop an advanced AI-based software tool capable of accurately detecting and classifying a range of real-world objects in Google Street View and 2D/3D views of Google Earth. The tool will be designed to identify "Generic Objects" such as cars, shops, trees, and other objects of interest, with the ultimate goal of providing relevant information to clients.\\

To ensure the tool's accuracy and effectiveness, the project will incorporate state-of-the-art computer vision and deep learning techniques. Literature surveys of existing research will be conducted to identify best practices and possible improvements. This will be followed by a milestone plan to guide the project's progress.\\

The tool's practical applications include urban planning, environmental monitoring, land use mapping, and disaster response. For example, it can assist in monitoring urban infrastructure, detecting changes in land use patterns, identifying areas affected by natural disasters, and even monitoring illegal activities such as logging or poaching.\\


\cleardoublepage
%\newpage
