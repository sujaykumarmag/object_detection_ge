%======================================================================
\chapter{Introduction}
%======================================================================





\section{Scope of the Project}
The project scope involves the development of an advanced AI-based software tool for object detection in Google Earth and Google Street View. The tool will be designed to identify and classify various real-world objects, including cars, shops, trees, and other objects of interest. The project will employ state-of-the-art computer vision and deep learning techniques to ensure the tool's accuracy and effectiveness.

The project view is to create a tool that can support various industries and domains, including urban planning, environmental monitoring, land use mapping, and disaster response. For example, the tool can assist in monitoring urban infrastructure, detecting changes in land use patterns, identifying areas affected by natural disasters, and even monitoring illegal activities such as logging or poaching. The tool will also enable clients to specify locations of interest and request data from Google API, which will be processed and provided to the client in a user-friendly format.

Google Earth Imagery can be used for a variety of purposes, such as urban planning, environmental monitoring, disaster response, and tourism. For example, urban planners can use the imagery to analyze land use patterns, identify areas for development, and assess the impact of new construction projects on the surrounding environment. Environmentalists can use the imagery to monitor changes in land cover, track deforestation rates, and identify areas affected by natural disasters such as fires, floods, or landslides.


%----------------------------------------------------------------------
\section{Google Earth Imagery}
%----------------------------------------------------------------------
Google Earth Imagery is a collection of high-resolution satellite and aerial images that provide a detailed view of the Earth's surface. This imagery is constantly updated and can be accessed through the Google Earth platform, allowing users to explore and navigate the planet from a bird's-eye perspective.

The images in Google Earth Imagery are typically captured by commercial satellite companies or government agencies, such as NASA or the US Geological Survey. These images are often taken using advanced sensors and cameras, which can capture fine details and subtle variations in the Earth's surface, such as terrain features, vegetation, and man-made structures.

Google Earth Imagery covers almost the entire planet and includes both rural and urban areas. The imagery is available in different resolutions, ranging from 15 meters per pixel (m/p) for some parts of the world, to as high as 15 cm/p for certain urban areas. The high-resolution imagery allows users to see details such as individual buildings, cars, and even people.

The imagery is also a popular tool for virtual tourism, allowing users to explore famous landmarks and tourist destinations around the world. Users can zoom in on specific locations, tilt and rotate the view, and even explore underwater areas through the use of 3D imagery.



\section{Constraints in Google Earth Imagery}
While Google Earth Imagery offers a wealth of data and information, there are several constraints and challenges that need to be considered. Here are some of the major ones:
\begin{enumerate}
    \item Resolution: The resolution of Google Earth imagery can vary widely, depending on the location and the type of imagery. In some cases, the resolution may not be high enough to detect small objects or features of interest.
    \item Cloud Cover: Cloud cover can obscure important details in Google Earth imagery, making it difficult to obtain accurate information about certain areas.
    \item Quality: Google Earth imagery quality can vary depending on the source and age of the imagery. In some cases, the imagery may be outdated or low-quality, which can affect the accuracy of the analysis.
    \item Image distortion: Imagery can be distorted due to the terrain and the angle of the camera when the image was captured. This can affect the accuracy of the analysis, especially when attempting to measure distances or identify specific features.
    \item Data Availability: Not all areas of the world have high-quality Google Earth imagery available. This can limit the scope of analysis for certain regions or countries.
    \item Data privacy: Google Earth imagery may capture sensitive or private information, such as military installations or private property. Careful consideration must be given to privacy concerns when using this data.
    \item Computational resources: Processing and analyzing large volumes of Google Earth imagery can be computationally intensive, requiring high-performance computing resources and specialized software tools.
\end{enumerate}
\section{Using ML Algorithms in Google Earth Imagery}
There are several potential use cases for using machine learning (ML) algorithms in Google Earth imagery to detect rooftops.
\begin{enumerate}
    \item Urban planning: City planners and developers can use ML algorithms to identify rooftops in satellite imagery to better understand the current urban landscape and to plan future developments.
    \item Disaster response: In the aftermath of natural disasters, such as earthquakes, floods, and wildfires, ML algorithms can be used to quickly identify damaged buildings and prioritize search and rescue efforts.
    \item Energy efficiency: ML algorithms can be used to identify rooftops that are suitable for solar panel installations, which can help to increase energy efficiency and reduce reliance on fossil fuels.
    \item Insurance: Insurance companies can use ML algorithms to assess the risk of damage to rooftops from severe weather events, which can help to inform pricing and coverage decisions.
    \item Property assessment: Real estate companies and property assessors can use ML algorithms to identify rooftops and assess the value of properties based on factors such as size, condition, and location.
\end{enumerate}
ML algorithms can help to automate the process of rooftop detection in Google Earth imagery, enabling faster and more accurate analysis of the data. This can lead to better decision-making in a variety of industries and domains.

