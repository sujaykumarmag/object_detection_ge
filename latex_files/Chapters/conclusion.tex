\chapter{Conclusions and Future Work}
\section{Conclusion}
we can conclude that deep learning techniques can effectively address the challenges of object detection in satellite imagery. The proposed method, which uses a deep learning model based on the Faster R-CNN architecture, achieved higher accuracy and faster processing time compared to other state-of-the-art object detection methods.

The proposed method has several potential applications, including urban planning, environmental monitoring, and disaster response. Future research could focus on improving the accuracy of the model by incorporating additional data sources, such as multi-spectral imagery, and exploring the use of transfer learning to reduce the need for large training datasets.

Overall, my work demonstrates the potential of deep learning techniques for object detection in satellite imagery and highlights the importance of developing new methods to improve the accuracy and efficiency of these techniques for real-world applications.

\section{Future Work}
As for the future work of the project, there are several potential directions that could be explored to further improve the accuracy and applicability of the proposed building detection method.

Firstly, incorporating additional data sources, such as multi-spectral or LiDAR data, could improve the accuracy of the detection method, especially in complex urban environments with high-rise buildings or dense vegetation.

Secondly, exploring transfer learning techniques could reduce the amount of labeled data required for training the model and allow for better generalization to different geographical areas or imaging conditions.

Finally, integrating the building detection method into a larger system for urban planning or disaster response could help to better understand and respond to urban changes and emergencies in real-time.